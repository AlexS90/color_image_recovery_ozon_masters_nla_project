\documentclass[12pt]{article}

\usepackage[T2A]{fontenc}
\usepackage[utf8]{inputenc}
\usepackage[russian]{babel}

\usepackage{amsmath}
\usepackage{geometry}
\usepackage[shortlabels]{enumitem}

\usepackage{hyphenat}
\usepackage[colorlinks=true, linkcolor=blue, urlcolor=blue, citecolor=blue, anchorcolor=blue]{hyperref}

\geometry{left=2cm, right=1cm, top=1.5cm, bottom=1.5cm}

% ================================================================


\begin{document}


\textbf{Название команды}:


\textbf{Состав команды}:
Донскова Мария, Кругликова Вероника, Серов Алексей


\textbf{Тема проекта}:
Восстановление цветных изображений с помощью низкоранговой  матрицы кватернионов.


\textbf{Описание проекта}:
Цель проекта~--- реализация эффективного алгоритма восстановления цветных изображений.
Простые методы низкоранговой аппроксимации матрицы с пропущенными значениями подходят только для одноканальных картинок (т.е. в градациях серого).
Если изображение цветное (то есть имеет 3 канала), то оно приводится каким-нибудь эвристическим методом к одноканальному (например, взвешенной суммой каналов).

Распространено мнение, что такой подход неоптимален, поскольку из взаимодействия цветов можно извлечь дополнительную информацию.
Однако непосредственная работа с трёхмерными тензорами ~--- это очень трудоёмкие вычисления, поскольку задача низкорангового приближения трёхмерного тензора является NP-полной.

Поэтому, чтобы работать с двумерными тензорами (то есть матрицами), мы будем представлять изображение в виде матрицы кватернионов.
Непосредственные вычисления с ними также сложны, но можно ввести взаимно обратное отображение матриц кватернионов в множество матриц комплексных чисел большего размера.
Схема восстановление изображений следующая: трёхканальная картинка с пропущенными пикселями $\mapsto$ матрица кватернионов $\mapsto$ матрица комплексных чисел $\mapsto$ низкоранговая аппроксимация $\mapsto$ восстановленная матрица кватернионов $\mapsto$ восстановленная картинка.  Результатом работы является реализация алгоритма восстановления цветных изображений с помощью матрицы кватернионов низкого ранга и сравнительный анализ с некоторыми известными алгоритмами.


\textbf{Список литературы}:
    \begin{enumerate}[label=\arabic*)]
	\item
	J. Miao, K. I. Kou, Color image recovery using low-rank quaternion matrix completion algorithm //arXiv preprint arXiv:1909.06567 (2019).\\
	URL:  \href{https://arxiv.org/abs/1909.06567}{https://arxiv.org/abs/1909.06567}

	\item
G. H. Golub, H. Gene, C. F. Van Loan,  Matrix Computations, The John Hopkins University Press Baltimore and London (1996).

	\item
	J. A. Bengua, H. N. Phien, H. D. Tuan, M. N. Do, Efficient tensor completion for color image and video recovery: Low-rank tensor train, IEEE Trans. Image Processing 26 (5) (2017) 2466–2479. \\
	URL:  \href{https://doi.org/10.1109/TIP.2017.2672439}{https://doi.org/10.1109/TIP.2017.2672439}
    \end{enumerate}

\end{document}
