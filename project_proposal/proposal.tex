\documentclass[12pt]{article}

\usepackage[T2A]{fontenc}
\usepackage[utf8]{inputenc}
\usepackage[russian]{babel}

\usepackage{amsmath}
\usepackage{geometry}
\usepackage[shortlabels]{enumitem}

\usepackage{hyphenat}
\usepackage[colorlinks=true, linkcolor=blue, urlcolor=blue, citecolor=blue, anchorcolor=blue]{hyperref}

\geometry{left=2cm, right=1cm, top=1.5cm, bottom=1.5cm}

% ================================================================


\begin{document}


\textbf{Название команды}:


\textbf{Состав команды}:
Донскова Мария, Кругликова Вероника, Серов Алексей


\textbf{Тема проекта}:
Восстановление цветных изображений с помощью низкоранговой аппроксимации матрицы кватернионов.


\textbf{Описание проекта}:
Цель проекта~--- реализация эффективного алгоритма восстановления цветных изображений.
Простые методы низкоранговой аппроксимации матрицы с пропущенными значениями подходят только для одноканальных картинок (т.е. в градациях серого).
Если изображение цветное (то есть имеет 3 канала), то оно приводится каким-нибудь эвристическим методом к одноканальному (например, взвешенной суммой каналов).

Распространено мнение, что такой подход неоптимален, поскольку из взаимодействия цветов можно извлечь дополнительную информацию.
Однако непосредственная работа с трёхмерными тензорами размера~--- это очень трудоёмкие вычисления, поскольку задача низкорангового приближения трёхмерного тензора является NP-полной (в отличие от низкорангового приближения матриц).

Поэтому, чтобы работать с двумерными тензорами (то есть матрицами), мы будем представлять изображение в виде матрицы кватернионов.
Непосредственные вычисления с ними также сложны, но можно ввести взаимно обратное отображение матриц кватернионов в множество матриц комплексных чисел большего размера.
Схема восстановление изображений следующая: трёхканальная картинка с пропущенными пикселями $\mapsto$ матрица кватернионов $\mapsto$ матрица комплексных чисел $\mapsto$ низкоранговая аппроксимация $\mapsto$ восстановленная матрица кватернионов $\mapsto$ восстановленная картинка.  


\textbf{Список литературы}:
    \begin{enumerate}[label=\arabic*)]
	\item
	    \href{https://arxiv.org/abs/1909.06567}{Статья}

	\item
	    Gene H. Golub, Charles F. Van Loan~--- Matrix Computations \textbf{НАПИСАТЬ НОРМАЛЬНУЮ ССЫЛКУ!}
    \end{enumerate}


\end{document}
